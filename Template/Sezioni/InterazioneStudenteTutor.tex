%----------------------------------------------------------------------------------------
%	DESCRIPTION OF THE INTERACTION BETWEEN THE STUDENT AND THE INTERNAL TUTOR
%----------------------------------------------------------------------------------------
\subsection{Interazione tra studente e tutor aziendale}
% Personalizzare definendo le modalità di interazione col tutor aziendale

A causa delle restrizioni nazionali per il distanziamento sociale dovuto al virus Covid19, lo stage è tenuto per la maggior parte del tempo da remoto, perciò non ci sarà una diretta interazione con il tutor aziendale. Tuttavia è fissato un incontro quotidiano a fine giornata, attraverso strumenti di comunicazione quali Skype e Microsoft Teams, per misurare i progressi ottenuti, al fine monitorare lo stato del progetto ma anche garantire il raggiungimento degli obiettivi fissati nella pianificazione. In questa occasione si potranno chiarire gli obiettivi da raggiungere e raffinare la pianificazione, qualora fosse necessario.
Infine è sempre possibile comunicare tramite e-mail o video-chiamate non pianificate per qualsiasi evenienza.

\subsubsection{Registro delle attività}
Per tracciare le attività svolte dallo studente durante lo stage è necessario riportare all'interno di un documento, condiviso con il tutor aziendale e con il tutor accademico, un riassunto di quanto svolto.
Lo strumento utilizzato è \textit{Google Drive} e al suo interno sarà presente una tabella contenente un riassunto chiaro e conciso delle attività svolte durante lo stage, riportate in ordine cronologico. \\
La registrazione delle attività viene eseguita rispettando le seguenti regole:
\begin{enumerate}
	\item Il documento deve essere aggiornato al termine di ogni giornata lavorativa;
	\item I dettagli del contenuto devono essere i seguenti:
		\begin{itemize}
			\item descrizione delle attività svolte;
			\item descrizione di eventuali problemi o impedimenti sorti.
		\end{itemize}
	\item Il tutor aziendale ed il tutor accademico potranno prendere visione del documento e apportare annotazioni in merito a mancanze o imprecisioni dello studente;
	\item Al termine dello stage, il tutor aziendale ed il tutor interno valideranno tale documento come garanzia dell'effettivo lavoro svolto.
\end{enumerate}