%----------------------------------------------------------------------------------------
%   USEFUL COMMANDS
%----------------------------------------------------------------------------------------

\newcommand{\dipartimento}{Dipartimento di Matematica ``Tullio Levi-Civita''}

%----------------------------------------------------------------------------------------
% 	USER DATA
%----------------------------------------------------------------------------------------

% Data di approvazione del piano da parte del tutor interno; nel formato GG Mese AAAA
% compilare inserendo al posto di GG 2 cifre per il giorno, e al posto di 
% AAAA 4 cifre per l'anno
\newcommand{\dataApprovazione}{Data}

% Dati dello Studente
\newcommand{\nomeStudente}{Massimo}
\newcommand{\cognomeStudente}{Toffoletto}
\newcommand{\matricolaStudente}{1161727}
\newcommand{\emailStudente}{massimo.toffoletto@studenti.unipd.it}
\newcommand{\telStudente}{+ 39 345 09 64 767}

% Dati del Tutor Aziendale
\newcommand{\nomeTutorAziendale}{Gregorio}
\newcommand{\cognomeTutorAziendale}{Piccoli}
\newcommand{\emailTutorAziendale}{gregorio.piccoli@zucchetti.it}
\newcommand{\telTutorAziendale}{+ 39 0371 59 457 24}
\newcommand{\ruoloTutorAziendale}{}

% Dati dell'Azienda
\newcommand{\ragioneSocAzienda}{Zucchetti S.p.A}
\newcommand{\indirizzoAzienda}{Via Giovanni Cittadella 765137, Padova (PD)}
\newcommand{\sitoAzienda}{https://www.zucchetti.it/}
\newcommand{\emailAzienda}{mail@mail.it}
\newcommand{\partitaIVAAzienda}{P.IVA 05006900962}

% Dati del Tutor Interno (Docente)
\newcommand{\titoloTutorInterno}{Prof.}
\newcommand{\nomeTutorInterno}{Paolo}
\newcommand{\cognomeTutorInterno}{Baldan}

\newcommand{\prospettoSettimanale}{
     % Personalizzare indicando in lista, i vari task settimana per settimana
     % sostituire a XX il totale ore della settimana
     \subsubsection{Prima Settimana}
    	\paragraph*{Studio di Google Assistant e implementazione di un relativo proof of concept} \mbox{}\\ [1mm]
    	Le attività che devono essere svolte sono le seguenti:
        \begin{itemize}
            \item Studio e analisi in autonomia di Google Assistant;
            \item Implementazione di un proof of concept sulla base di quanto analizzato.
        \end{itemize}
    	Nel corso della prima settimana è previsto uno studio approfondito e analitico della documentazione di Google Assistant, prestando particolare attenzione al modo in cui si implementano intent e skill, alla struttura del prodotto software, al modo in cui avviene l'interazione sia lato utente che lato server e alla sicurezza dei dati durante tutta la comunicazione. È quindi necessario comprendere e riportare quali funzionalità sono permesse e messe a disposizione e quali invece sono i limiti di tale tecnologia.
    	Concorrentemente a questa attività, è previsto lo svolgimento del proof of concept che presenta una skill o un intent di esempio illustrativo per dare concretezza allo studio fatto. 
    \subsubsection{Seconda Settimana}
    	\paragraph*{Studio di Apple Siri e implementazione di un relativo proof of concept} \mbox{}\\ [1mm]
    	Le attività che devono essere svolte sono le seguenti:
        \begin{itemize}
            \item Studio e analisi in autonomia di Apple Siri;
            \item Implementazione di un proof of concept sulla base di quanto analizzato.
        \end{itemize}
	    Nel corso della seconda settimana è previsto uno studio approfondito e analitico della documentazione di Apple Siri, prestando particolare attenzione al modo in cui si implementano intent e skill, alla struttura del prodotto software, al modo in cui avviene l'interazione sia lato utente che lato server e alla sicurezza dei dati durante tutta la comunicazione. È quindi necessario comprendere e riportare quali funzionalità sono permesse e messe a disposizione e quali invece sono i limiti di tale tecnologia.
	    Concorrentemente a questa attività, è previsto lo svolgimento del proof of concept che presenta una skill o un intent di esempio illustrativo per dare concretezza allo studio fatto.
	\subsubsection{Terza Settimana}
		\paragraph*{Studio di Amazon Alexa e implementazione di un relativo proof of concept} \mbox{}\\ [1mm]
	   	Le attività che devono essere svolte sono le seguenti:
        \begin{itemize}
            \item Studio e analisi in autonomia di Amazon Alexa;
            \item Implementazione di un proof of concept sulla base di quanto analizzato.
        \end{itemize}
	    Nel corso della terza settimana è previsto uno studio approfondito e analitico della documentazione di Amazon Alexa, prestando particolare attenzione al modo in cui si implementano intent e skill, alla struttura del prodotto software, al modo in cui avviene l'interazione sia lato utente che lato server e alla sicurezza dei dati durante tutta la comunicazione. È quindi necessario comprendere e riportare quali funzionalità sono permesse e messe a disposizione e quali invece sono i limiti di tale tecnologia.
	    Concorrentemente a questa attività, è previsto lo svolgimento del proof of concept che presenta una skill o un intent di esempio illustrativo per dare concretezza allo studio fatto.
    \subsubsection{Quarta Settimana}
    	\paragraph*{Test e documentazione comparativa di quanto svolto nelle settimane precedenti} \mbox{}\\ [1mm]
    	Le attività che devono essere svolte sono le seguenti:
        \begin{itemize}
            \item Test sui proof of concept implementati;
            \item Documentazione e comparazione approfondita sulle tecnologie analizzate.
        \end{itemize}
    	Nel corso della quarta settimana è previsto lo svolgimento dei test su quanto realizzato fino a questo momento per verificarne il corretto funzionamento. Inoltre viene svolta un'ulteriore attività di documentazione atta a riportare in modo dettagliato la comparazione tra le tecnologie studiate nelle settimane precedenti.
    \subsubsection{Quinta Settimana}
    	\paragraph*{Apprendimento del sistema Zucchetti per l'interazione coi comandi vocali} \mbox{}\\ [1mm]
    	Le attività che devono essere svolte sono le seguenti:
        \begin{itemize}
            \item Apprendimento del funzionamento del sistema Zucchetti;
            \item Apprendimento della realizzazione di grammatiche basate sull'algoritmo HPP.
        \end{itemize}
    	Nel corso della quinta settimana è previsto l'apprendimento del sistema Zucchetti con cui deve avvenire l'integrazione di quanto analizzato precedentemente. In particolare si deve studiare e capire l'algoritmo HPP di Zucchetti basato sul parsing e su modelli matematici di tipo probabilistico, che riconosce e comprende le frasi pronunciate da una persona fisica. Esso inoltre si basa sull'idea seguente: è importante avere un altissima precisione nel riconoscimento dei comandi perché essi possono riguardare azioni critiche, anche a scapito di un'elevata recall, ovvero il mancato riconoscimento di comandi corretti. Perciò l'obiettivo è anche capire come perseguire tale concetto.
     \subsubsection{Sesta Settimana}
     	\paragraph*{Realizzazione di una grammatica che implementi determinate skill del prodotto} \mbox{}\\ [1mm]
    	Le attività che devono essere svolte sono le seguenti:
        \begin{itemize}
            \item Unione delle conoscenze apprese sulle tecnologie e sul sistema Zucchetti;
            \item Realizzazione di una grammatica in grado di implementare una determinata skill o intent.
        \end{itemize}
    	Nel corso della sesta settimana è prevista un'attività di integrazione tra le conoscenze acquisite fino a questo momento e di successiva applicazione. È quindi  implementata una grammatica capace di comprendere determinati comandi e svolgere azioni ad essi correlate; la grammatica sarà integrata nel sistema Zucchetti. Per fare ciò, è necessario tenere in considerazione i vantaggi e gli svantaggi delle tecnologie analizzate per ottenere il risultato migliore possibile.
    \subsubsection{Settima Settimana}
    	\paragraph*{Apprendimento autonomo del sistema durante l'esecuzione} \mbox{}\\ [1mm]
    	Le attività che devono essere svolte sono le seguenti:
        \begin{itemize}
            \item Implementazione della capacità di apprendere nuove frasi del linguaggio naturale run time.
        \end{itemize}
    	Durante la settima settimana è prevista l'implementazione dell'addestramento del sistema durante la sua esecuzione. Più in dettaglio si deve implementare la "learning mode": mentre il sistema è in esecuzione, qualora un utente impartisca un comando non pienamente compreso, esso può attivare una funzione di addestramento che gli permette di chiarire quanto richiesto, interagendo con l'utente finale, e imparare nuove espressioni del linguaggio naturale. 
    \subsubsection{Ottava Settimana}
    	\paragraph*{Test e documentazione di quanto svolto nelle settimane precedenti} \mbox{}\\ [1mm]
    	Le attività che devono essere svolte sono le seguenti:
        \begin{itemize}
        	\item Test sull'integrazione finale delle funzionalità di apprendimento e addestramento implementate;
        	\item Documentazione del codice prodotto e delle tecnologie analizzate.
        \end{itemize}
        Nel corso dell'ottava settimana è previsto lo svolgimento dei test finali sul prodotto software realizzato per verificarne il corretto funzionamento. Inoltre viene svolta un'attività di documentazione della codifica ma, più in generale, di tutto il lavoro svolto sul prodotto software.
}

% Indicare il totale complessivo (deve essere compreso tra le 300 e le 320 ore)
\newcommand{\totaleOre}{}

\newcommand{\obiettiviObbligatori}{
	 \item \underline{\textit{O01}}: primo obiettivo;
	 \item \underline{\textit{O02}}: secondo obiettivo;
	 \item \underline{\textit{O03}}: terzo obiettivo;
	 
}

\newcommand{\obiettiviDesiderabili}{
	 \item \underline{\textit{D01}}: primo obiettivo;
	 \item \underline{\textit{D02}}: secondo obiettivo;
}

\newcommand{\obiettiviFacoltativi}{
	 \item \underline{\textit{F01}}: primo obiettivo;
	 \item \underline{\textit{F02}}: secondo obiettivo;
	 \item \underline{\textit{F03}}: terzo obiettivo;
}