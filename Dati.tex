%----------------------------------------------------------------------------------------
%   USEFUL COMMANDS
%----------------------------------------------------------------------------------------

\newcommand{\dipartimento}{Dipartimento di Matematica ``Tullio Levi-Civita''}

%----------------------------------------------------------------------------------------
% 	USER DATA
%----------------------------------------------------------------------------------------

% Data di approvazione del piano da parte del tutor interno; nel formato GG Mese AAAA
% compilare inserendo al posto di GG 2 cifre per il giorno, e al posto di 
% AAAA 4 cifre per l'anno
\newcommand{\dataApprovazione}{Data}

% Dati dello Studente
\newcommand{\nomeStudente}{Massimo}
\newcommand{\cognomeStudente}{Toffoletto}
\newcommand{\matricolaStudente}{1161727}
\newcommand{\emailStudente}{massimo.toffoletto@studenti.unipd.it}
\newcommand{\telStudente}{+ 39 345 09 64 767}

% Dati del Tutor Aziendale
\newcommand{\nomeTutorAziendale}{Gregorio}
\newcommand{\cognomeTutorAziendale}{Piccoli}
\newcommand{\emailTutorAziendale}{gregorio.piccoli@zucchetti.it}
\newcommand{\telTutorAziendale}{+ 39 0371 59 457 24}
\newcommand{\ruoloTutorAziendale}{}

% Dati dell'Azienda
\newcommand{\ragioneSocAzienda}{Zucchetti S.p.A}
\newcommand{\indirizzoAzienda}{Via Giovanni Cittadella 765137, Padova (PD)}
\newcommand{\sitoAzienda}{https://www.zucchetti.it/}
\newcommand{\emailAzienda}{mail@mail.it}
\newcommand{\partitaIVAAzienda}{P.IVA 05006900962}

% Dati del Tutor Interno (Docente)
\newcommand{\titoloTutorInterno}{Prof.}
\newcommand{\nomeTutorInterno}{Paolo}
\newcommand{\cognomeTutorInterno}{Baldan}

\newcommand{\prospettoSettimanale}{
     % Personalizzare indicando in lista, i vari task settimana per settimana
     % sostituire a XX il totale ore della settimana
     \subsubsection{Prima Settimana}
    	\paragraph{Studio di Google Assistant e implementazione di un relativo proof of concept}
    	\paragraph*{Obiettivi} \mbox{}\\ [1mm]
    	Gli obiettivi fissati in questo periodo sono i seguenti:
        \begin{itemize}
            \item \textbf{OB-1}: studio e analisi di Google Assistant;
            \item \textbf{OB-2}: implementazione di una skill basata su un intent per Google Assistant, sottoforma di proof of concept.
        \end{itemize}
	    \paragraph*{Descrizione} \mbox{}\\ [1mm]
    	Nel corso della prima settimana è previsto uno studio approfondito e analitico della documentazione di Google Assistant, prestando particolare attenzione al modo in cui si implementano intent per le skill, alla struttura del prodotto software, al modo in cui avviene l'interazione lato utente e lato server e alla sicurezza dei dati durante l'intera comunicazione. È quindi necessario comprendere e riportare quali funzionalità sono messe a disposizione e quali sono invece i limiti di tale tecnologia. \\
    	Concorrentemente a questa attività, è previsto lo sviluppo di un proof of concept che implementi una skill con un intent di esempio per dare concretezza allo studio fatto. 
    \subsubsection{Seconda Settimana}
    	\paragraph{Studio di Amazon Alexa e implementazione di un relativo proof of concept}
    	\paragraph*{Obiettivi} \mbox{}\\ [1mm]
    	Gli obiettivi fissati in questo periodo sono i seguenti:
    	\begin{itemize}
    		\item \textbf{OB-3}: studio e analisi di Amazon Alexa;
    		\item \textbf{OB-4}: implementazione di una skill basata su un intent per Amazon Alexa, sottoforma di proof of concept.
    	\end{itemize}
        \paragraph*{Descrizione} \mbox{}\\ [1mm]
    	Nel corso della seconda settimana è previsto uno studio approfondito e analitico della documentazione di Amazon Alexa, prestando particolare attenzione al modo in cui si implementano intent per le skill, alla struttura del prodotto software, al modo in cui avviene l'interazione lato utente e lato server e alla sicurezza dei dati durante l'intera comunicazione. È quindi necessario comprendere e riportare quali funzionalità sono messe a disposizione e quali sono invece i limiti di tale tecnologia. \\
    	Concorrentemente a questa attività, è previsto lo sviluppo di un proof of concept che implementi una skill con un intent di esempio per dare concretezza allo studio fatto.
	\subsubsection{Terza Settimana}
		\paragraph{Studio di Apple Siri e implementazione di un relativo proof of concept}
		\paragraph*{Obiettivi} \mbox{}\\ [1mm]
		Gli obiettivi fissati in questo periodo sono i seguenti:
		\begin{itemize}
			\item \textbf{OF-1}: studio e analisi di Apple Siri;
			\item \textbf{OF-2}: implementazione di una skill basata su intent per Apple Siri, sottoforma di proof of concept.
		\end{itemize}
		Questi obiettivi sono facoltativi in quanto è necessaria la presenza in azienda per poterli perseguire ma, a causa del distanziamento sociale dovuto allo stato pandemico attuale, non è garantita. Si prevede dunque una pianificazione alternativa data dall'approfondimento dei seguenti obiettivi:
		\begin{itemize}
			\item \textbf{OB-2}: implementazione di una skill basata su un intent per Google Assistant, sottoforma di proof of concept;
			\item \textbf{OB-4}: implementazione di una skill basata su un intent per Amazon Alexa, sottoforma di proof of concept.
		\end{itemize}
		\paragraph*{Descrizione} \mbox{}\\ [1mm]
		Nel corso della terza settimana è previsto uno studio approfondito e analitico della documentazione di Apple Siri, prestando particolare attenzione al modo in cui si implementano intent per le skill, alla struttura del prodotto software, al modo in cui avviene l'interazione lato utente e lato server e alla sicurezza dei dati durante l'intera comunicazione. È quindi necessario comprendere e riportare quali funzionalità messe a disposizione e quali sono invece i limiti di tale tecnologia. \\
		Concorrentemente a questa attività, è previsto lo sviluppo di un proof of concept che implementi una skill con un intent di esempio per dare concretezza allo studio fatto. \\
		Secondo la pianificazione alternativa, invece, è previsto un miglioramento dei proof of concept precedentemente realizzati, implementando slot, utterances ed eventualmente più di una skill.
    \subsubsection{Quarta Settimana}
    	\paragraph{Test e documentazione comparativa di quanto svolto nelle settimane precedenti}
    	\paragraph*{Obiettivi} \mbox{}\\ [1mm]
    	L'obiettivo fissato in questo periodo è il seguente:
        \begin{itemize}
            \item \textbf{OB-5}: stesura analisi comparativa delle tecnologie studiate.
        \end{itemize}
    	Inoltre devono essere implementati i test per verificare la correttezza del prodotto.
	    \paragraph*{Descrizione} \mbox{}\\ [1mm]
    	Nel corso della quarta settimana è previsto lo svolgimento dei test su quanto realizzato, per verificarne il corretto funzionamento. Successivamente verrà svolta un'attività di documentazione atta a riportare in modo dettagliato la comparazione tra le tecnologie studiate nelle settimane precedenti.
    \subsubsection{Quinta Settimana}
    	\paragraph{Apprendimento del sistema Zucchetti per l'interazione coi comandi vocali}
    	\paragraph*{Obiettivi} \mbox{}\\ [1mm]
    	L'obiettivo fissato in questo periodo è il seguente:
        \begin{itemize}
            \item \textbf{OB-6}: studio dell'algoritmo HPP (Hidden Probabilistic Parser).
        \end{itemize}
	    \paragraph*{Descrizione} \mbox{}\\ [1mm]
    	Nel corso della quinta settimana è previsto l'apprendimento del sistema Zucchetti con cui deve avvenire l'integrazione di quanto analizzato precedentemente. In particolare si deve studiare e capire l'algoritmo HPP di Zucchetti basato sul parsing e su modelli matematici di tipo probabilistico. Esso è in grado di riconoscere e comprendere le frasi pronunciate da una persona fisica.
     \subsubsection{Sesta Settimana}
     	\paragraph{Realizzazione di una grammatica che implementi determinate skill del prodotto}
     	\paragraph*{Obiettivi} \mbox{}\\ [1mm]
		L'obiettivo fissato in questo periodo è il seguente:
        \begin{itemize}
            \item \textbf{OB-7}: realizzazione di una grammatica che implementi una skill non banale, integrata al sistema Zucchetti esistente.
        \end{itemize}
	    \paragraph*{Descrizione} \mbox{}\\ [1mm]
    	Nel corso della sesta settimana è prevista un'attività di integrazione tra le conoscenze acquisite fino a questo momento e di successiva applicazione. È quindi implementata una grammatica capace di comprendere determinati comandi e svolgere azioni ad essi correlate; la grammatica sarà integrata nel sistema Zucchetti. Per fare ciò, è necessario tenere in considerazione i vantaggi e gli svantaggi delle tecnologie analizzate per ottenere il risultato migliore possibile.
    \subsubsection{Settima Settimana}
    	\paragraph{Apprendimento autonomo del sistema durante l'esecuzione}
    	\paragraph*{Obiettivi} \mbox{}\\ [1mm]
    	L'obiettivo fissato in questo periodo è il seguente:
        \begin{itemize}
            \item \textbf{OD-01}: implementazione dell'apprendimento autonomo legato alla skill realizzata del sistema di riconoscimento vocale, quando esso è in esecuzione.
        \end{itemize}
	    \paragraph*{Descrizione} \mbox{}\\ [1mm]
    	Nel corso della settima settimana è prevista l'implementazione dell'addestramento del sistema durante la sua esecuzione. Più in dettaglio si deve implementare la "learning mode": mentre il sistema è in esecuzione, qualora un utente impartisca un comando non pienamente compreso, esso può attivare una funzione di addestramento che gli permette di chiarire quanto richiesto, interagendo con l'utente finale, e imparare nuove espressioni del linguaggio naturale. 
    \subsubsection{Ottava Settimana}
    	\paragraph{Test e documentazione di quanto svolto nelle settimane precedenti}
    	\paragraph*{Obiettivi} \mbox{}\\ [1mm]
    	In questo ultimo periodo è prevista l'implementazione dei test e la stesura della documentazione finale sul prodotto realizzato.
	    \paragraph*{Descrizione} \mbox{}\\ [1mm]
        Nel corso dell'ottava settimana è previsto lo svolgimento dei test finali sul prodotto software realizzato per verificarne il corretto funzionamento. Inoltre viene svolta un'attività di documentazione della codifica ma, più in generale, di tutto il lavoro svolto sul prodotto software.
}

% Indicare il totale complessivo (deve essere compreso tra le 300 e le 320 ore)
\newcommand{\totaleOre}{}

\newcommand{\obiettiviObbligatori}{
	 \item \textbf{OB-1}: studio e analisi di Google Assistant;
	 \item \textbf{OB-2}: implementazione di una skill basata su un intent per Google Assistant, sottoforma di proof of concept;
 	 \item \textbf{OB-3}: studio e analisi di Amazon Alexa;
	 \item \textbf{OB-4}: implementazione di una skill basata su un intent per Amazon Alexa, sottoforma di proof of concept; 
	 \item \textbf{OB-5}: stesura analisi comparativa delle tecnologie studiate;
	 \item \textbf{OB-6}: studio dell'algoritmo HPP (Hidden Probabilistic Parser);
	 \item \textbf{OB-7}: realizzazione di una grammatica che implementi una skill non banale, integrata al sistema Zucchetti esistente.
}

\newcommand{\obiettiviDesiderabili}{
	 \item \textbf{OD-01}: implementazione dell'apprendimento autonomo legato alla skill realizzata del sistema di riconoscimento vocale, quando esso è in esecuzione.
}

\newcommand{\obiettiviFacoltativi}{
 	 \item \textbf{OF-1}: studio e analisi di Apple Siri;
	 \item \textbf{OF-2}: implementazione di una skill basata su intent per Apple Siri, sottoforma di proof of concept;
}